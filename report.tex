\documentclass[12pt,lettersize]{article}
\usepackage{amsmath}
\usepackage{amsfonts}
\usepackage{xargs}
\usepackage{color}
\usepackage{amssymb}
\usepackage{graphicx}
\usepackage{listings}
\usepackage{tabu}
\usepackage{hyperref}
\usepackage{todonotes}

\newcommandx{\lef}[2][1=]{\todo[linecolor=red,backgroundcolor=red!25,bordercolor=red,size=\footnotesize,#1]{LEF: #2}}
\newcommand{\mwand}{\mathrel{-\mkern-6mu*}}
\newcommand{\valid}[1]{\texttt{valid}\ #1}

\usepackage{listings,lstlangcoq}
\lstdefinestyle{coq}{
  columns=flexible,
  mathescape=true,
  belowcaptionskip=1\baselineskip,
  breaklines=true,
  xleftmargin=0pt,
  language=Coq,
  morekeywords={Variant, fun, Arguments, Type, cofix},
  % morekeywords={SOCKAPI,ITREE,data_at,data_at_},
  emph={%
    SOCKAPI,ITree,CTree,data_at,data_at_
  },
  emphstyle={\bfseries\color{green!40!red!80}},
  showstringspaces=false,
  basicstyle=\small\ttfamily,
  keywordstyle=\bfseries\color{green!20!black},
  commentstyle=\itshape\color{red!40!black},
  identifierstyle=\color{violet!50!black},
  stringstyle=\color{orange},
  escapeinside={<@}{@>}
}
\newcommand{\inlinecoq}[1]{\mbox{\lstinline[style=coq,columns=fixed,basewidth=.48em]{#1}}}


\title{Report: Automated simplification tactics over separation algebras.}
\author{Lef Ioannidis, Yiyun Liu}
\date{\today}

\begin{document}
\maketitle


\begin{abstract}
This project proposes the development of automated reasoning tactics for Separation Logic in Coq's Ltac2, based on the \emph{separata}
and \emph{starforce} tactics~\cite{hou2017proof} in Isabel/HOL. The proposed automation will be
abstract over Separation Algebras (SAs)~\cite{calcagno2007local}, to ensure general applicability to programs with 
ghost state and different shared resources, and will leverage automation techniques like proof-by-reflection.
The overall goal is to partially automate the verification process of Separation Logic programs by building
a more efficient and general version of \texttt{xsimpl}~\cite{chargueraud2020separation}.
\end{abstract}

\section{Background and Motivation}
Separation logic significantly simplifies reasoning about programs that manipulate heap-allocated memory, offering a framework
to ensure the correctness of these operations. The tactics introduced in "Separation Logic Foundations"~\cite{chargueraud2020separation}
are especially effective, although their implementation is "fairly involved", according to their author. The \texttt{xsimpl} tactic
for automatically solving assertions is particularly ad-hoc, defined as a functor over many separation logic primitives and laws,
making it cumbersome to initialize.

A more structured and general altenative by Hou et al~\cite{hou2017proof} for Isabelle/HOL, provides powerful automated reasoning
capabilities within Separation Logics. Adapting these tactics --- specifically \emph{separata} and 
\emph{starforce} --- for Coq's Ltac2 language can greatly enhance the separation logic verification process in Coq.
By abstracting these tactics over SAs~\cite{calcagno2007local} we aim to offer
flexible tactics that can be applied to programs with ghost state, permissions~\cite{bornat2005permission}, 
histories~\cite{sergey2015specifying} and more.

\section{Project Goal}
Our goal is to implement the \emph{separata} and \emph{starforce} tactics for separation logic in Coq's Ltac2, abstracting over
a Partial Commutative Monoid (PCM) structure modeling the shared resources at hand. This implementation will not only replicate
the functionalities outlined in the referenced paper but also extend them with techniques
like \emph{proof-by-reflection}.

We will evaluate the success of this project by semi-automating the proving of two examples over at two different SAs.

\section{Deliverables}
This project will result in several key components that improve the automated reasoning capabilities of the Coq proof assistant in 
separation logics. The core deliverables include:

\begin{itemize}
\item \textbf{Separata Tactic:} A fully implemented \emph{separata} tactic within Coq's Ltac2 language, enabling automated
reasoning over separation logic assertions. The \emph{separata} tactic is based on an embedding of a labelled sequent calculus
and is designed to work with abstract separation logic~\cite{hou2017proof}.

\item \textbf{Starforce tactic:} Also implemented in Ltac2, equipped with specialized proof search strategies for efficiently handling
separating conjunctions and magic wands. This tactic will further automate the reasoning process with more specialized
automation.

\item \textbf{PCM Algebraic Structure:} An algebraic structure for Partial Commutative Monoids (PCMs).
This structure will ensure that the tactics are not only applicable to heaps, but a wide range of separation algebras as well.

\item \textbf{Examples:} A collection of examples demonstrating the use of the implemented tactics in automating the proof of properties
across two different separation algebras. These examples will showcase the power and flexibility of the tactics in realistic 
formal verification use-cases.
\end{itemize}

\bibliographystyle{plain}
\bibliography{main}

\end{document}
